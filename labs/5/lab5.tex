\documentclass{article}

\usepackage{siunitx} % Provides the \SI{}{} and \si{} command for typesetting SI units

\usepackage{graphicx} % Required for the inclusion of images

\usepackage{tabularx} % Tables

\usepackage{natbib} % Required to change bibliography style to APA

\usepackage{amsmath} % Required for some math elements

\usepackage{parskip} % Formatting

\usepackage{tikz}
\usetikzlibrary{automata,positioning} %for nodes

% Many options for pseudocode
\usepackage{algorithm2e}
\usepackage{algorithmic}

\usepackage[demo]{graphicx}
\usepackage{caption}
\usepackage{subcaption}

\usepackage{listings} % Python code blocks

\setlength\parindent{0pt} % Removes all indentation from paragraphs

\title{Lab 5 Report: Localization} % Title

\author{Team 4 \\\\ Rohan Wagh \\ Oliver Rayner \\ Gabriel Jimenez \\ Kairo Morton \\\\ 6.4200/16.405: Robotics Science and Systems} % Team # + Names, Class (RSS)

\date{\today} % Date for the report

\begin{document}

\maketitle

\section{Motion Model}
In order to accurately determine the position of the robot in a known map motion data must be incorporated. This data will allow the algorithm to update the empirical distribution over the robot's pose represented by the set of particles. In order to update this distribution we treat the wheel odometry data as consisting of three random variables $\Delta x, \Delta y,  \Delta \theta$. The robot provides the mean values for these random variables as well as a 3x3 covariance matrix to model measurement uncertainty. We assume the 2D odometry of the robot ($\Delta x, \Delta y$) can be modeled by a 2-dimensional multivariate Gaussian distribution using the upper 2x2 corner of the 3x3 covariance matrix. We also assume the independence of $\Delta \theta$ as it relates to the other odometry data and as such we model $\Delta \theta$ using a VonMises distribution with a mean of the measured $\Delta \theta$ and a concentration, $\kappa$, equal to the inverse of the variance of $\Delta \theta$ retrieved from the 3x3 covariance matrix. The VonMises distribution was chosen due to its similarity to the Gaussian distribution while still being circular, thus, giving accurate probability values for all angles. 

Using the definitions above the motion model is implemented simply by sampling a new  $\Delta x^\prime, \Delta y^\prime,  \Delta \theta^\prime$ for each point from the distributions specified above. This new sampled odometry data, $\Delta \textbf{x}^\prime$, which accounts for uncertainty is then used to update that particle's using the following equation:

$$\textbf{x}_{\text{new}} = \begin{bmatrix}
\cos(-\theta_{\text{old}}) & -\sin(-\theta_{\text{old}}) & 0\\ 
\sin(-\theta_{\text{old}}) & \cos(-\theta_{\text{old}}) & 0\\ 
0 & 0 & 1
\end{bmatrix}\Delta \textbf{x}^\prime + \textbf{x}_{\text{old}}$$

The new particles represent the distribution over robot poses after accounting for the most recent motion of the robot and uncertainty.

\section{Lessons Learned}

\subsection{Kairo Morton}
From my perspective, the lab 5 assignment was a success. I found my section, which involved writing the motion model, to be straightforward, and I was able to complete it with ease. Additionally, I enjoyed assisting Gabriel in debugging the sensor model, which allowed me to contribute to the team's success beyond my assigned tasks. Finally, clear communication between team members played a critical role in making the integration of our work mostly seamless. This experience has continue to show me the importance of effective communication in any collaborative project, particularly those that involve complex robotics tasks such as localization with many components. Overall, I am proud of what our team accomplished and believe we worked well together to achieve our objectives.




\end{document}
